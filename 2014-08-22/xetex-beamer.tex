\documentclass[notheorems,mathserif,table,compress]{beamer}  %dvipdfm选项是关键,否则编译统统通不过
%%------------------------常用宏包------------------------
%%注意, beamer 会默认使用下列宏包: amsthm, graphicx, hyperref, color, xcolor, 等等
\usepackage{fontspec,xunicode,xltxtra}  % for XeTeX
\usepackage{comment}
\usepackage{fancybox}
%\usepackage{iplouclistings}
\usepackage{styles/iplouccfg}
\usepackage{styles/zhfontcfg}
\usepackage{styles/iplouclistings}

%%------------------------ThemeColorFont------------------------
%% Presentation Themes
% \usetheme[<options>]{<name list>}
\usetheme{Madrid}
%% Inner Themes
% \useinnertheme[<options>]{<name>}
%% Outer Themes
% \useoutertheme[<options>]{<name>}
\useoutertheme{miniframes} 
%% Color Themes 
% \usecolortheme[<options>]{<name list>}
%% Font Themes
% \usefonttheme[<options>]{<name>}
\setbeamertemplate{background canvas}[vertical shading][bottom=white,top=structure.fg!7] %%背景色, 上25%的蓝, 过渡到下白.
\setbeamertemplate{theorems}[numbered]
\setbeamertemplate{navigation symbols}{}   %% 去掉页面下方默认的导航条.
%\usepackage{zhfontcfg}
%\setsansfont[Mapping=tex-text]{文泉驿正黑}  %% 需要fontspec宏包
     %如果装了Adobe Acrobat,可在font.conf中配置Adobe字体的路径以使用其中文字体
     %也可直接使用系统中的中文字体如SimSun,SimHei,微软雅黑 等
     %原来beamer用的字体是sans family;注意Mapping的大小写,不能写错
     %设置字体时也可以直接用字体名,以下三种方式等同:
     %\setromanfont[BoldFont={黑体}]{宋体}
     %\setromanfont[BoldFont={SimHei}]{SimSun}
     %\setromanfont[BoldFont={"[simhei.ttf]"}]{"[simsun.ttc]"}
%%------------------------MISC------------------------
\graphicspath{{figures/}}         %% 图片路径. 本文的图片都放在这个文件夹里了.
%%------------------------正文------------------------
\begin{document}
\XeTeXlinebreaklocale "zh"         % 表示用中文的断行
\XeTeXlinebreakskip = 0pt plus 1pt % 多一点调整的空间
%%----------------------------------------------------------
%% This is only inserted into the PDF information catalog. Can be left
%% out.
%%%
%% Delete this, if you do not want the table of contents to pop up at
%% the beginning of each subsection:
\begin{comment}
\AtBeginSection[]{                              % 在每个Section前都会加入的Frame
  \frame<handout:0>{
    \frametitle{Content}\small
    \tableofcontents[current,currentsubsection]
  }
}
\AtBeginSubsection[]                            % 在每个子段落之前
{
  \frame<handout:0>                             % handout:0 表示只在手稿中出现
  {
    \frametitle{下一节内容}\small
    \tableofcontents[current,currentsubsection] % 显示在目录中加亮的当前章节
  }
}
\end{comment}
%%----------------------------------------------------------
\title[]{Draw textures on scallop shells}
\author[Xue Sun]{\textcolor{olive}{Xue Sun}}
  %\hspace{2.28em}导师~~\textcolor{olive}{姬光荣}~教授}
\institute[Ocean University of China]{\small\textcolor{violet}{Ocean University of China}}
\date{August 8th}
%\titlegraphic{\vspace{-6em}\includegraphics[height=7cm]{ouc}\vspace{-6em}}
\frame{ \titlepage }
%%----------------------------------------------------------
%\section*{目录}
\frame{\frametitle{content}\tableofcontents}
%%----------------------------------------------------------

%\section{Beamer类和XeTeX概览} %如果你想书签不出现问题,请不要用\XeTeX
                                 %这类复杂的指令,直接写XeTeX吧
\section{Introduction}
\begin{frame}
  \frametitle{Introduction}
  Key step: extract textures on scallop shells. 
\end{frame}
\section{Draw radial ribs}
\begin{frame}
  \frametitle{Draw radial ribs}
  Steps:
  \begin{itemize}
  \item Strengthen textures.
  \item Apply hough transformation.
  \item Connect fracture lines.
  \item Draw undetected lines.
  \end{itemize}
\end{frame}

%\section{用XeTeX 和beamer包制作演示文稿(幻灯片)}
\begin{frame}
  \frametitle{Strengthen textures.}
  \begin{figure}[!ht]
  \begin{minipage}[t]{0.4\textwidth}
  \centering
  \includegraphics[width=1.5in]{1605.png}
  \end{minipage}
  \begin{minipage}[t]{0.4\textwidth}
  \includegraphics[width=1.5in]{1605_parameter5.png}
  \end{minipage}
  \caption{Result of strengthen textures.}
  \end{figure} 
\end{frame}

\begin{frame}
  \frametitle{Apply hough transformation and connect fracture lines.}
  \begin{figure}[!ht]
  \begin{minipage}[t]{0.4\textwidth}
  \centering
  \includegraphics[width=2in]{fracture-lines.png}
  \end{minipage}
  \begin{minipage}[t]{0.4\textwidth}
  \includegraphics[width=2in]{connect-lines.png}
  \end{minipage}
  \caption{Fracture lines and connected lines.}
  \end{figure} 
\end{frame}

\begin{frame}
  \frametitle{Add undetected lines}
  \begin{figure}[!ht]
  \centering\includegraphics[width=2in]{connect-lines.png}
  \end{figure} 
  Angles between lines:
  \begin{table}
  \begin{tabular}{|c|c|c|c|c|c|c|c|c|c|c|}
  \hline
  4&5&5.5&5.5&5.5&7.5&5.5&5&6&7&6\\
  \hline
  \end{tabular}
  \end{table}
 Mean: 5.5
\end{frame}

\begin{frame}
  \frametitle{Draw growth lines}
   Curve fitting matlab tool:
  \begin{figure}[!ht]
  \centering\includegraphics[width=3.5in]{curve-fitting.png}
  \end{figure} 
\end{frame}

\section{Draw growth lines}
\begin{frame}
  \frametitle{Draw growth lines}
   Result of curve fitting:
  \begin{figure}[!ht]
  \centering\includegraphics[width=2.5in]{growth-line.png}
  \end{figure} 
\end{frame}

\begin{frame}
  \frametitle{Target of next week.}
  Try to realize the drawing of growth lines and radial ribs.
\end{frame}

\begin{frame}
\begin{beamercolorbox}{red}
\Ovalbox{a}
\end{beamercolorbox}%
\end{frame}
\begin{frame}

\end{frame}
\begin{bash}
#!/bin/bash
if [ $# == 1 ]; then
    echo -ne "Deleting FILES including [$1] in the CURRENT directory ...\n\n"
    for i in $(tree -a -f -i | grep "$1")
    do
      echo -ne "Deleting $i\n"
      rm -f $i
    done
elif [ $# == 2 ]; then
    echo -ne "Deleting FILES including [$1] in [$2] directory ...\n"
    for i in $(tree -a -f -i $2 |grep "$1")
    do
      echo -ne "Deleting $i\n"
      rm -f $i
    done    
else
    echo -ne "Arguments Error.\n"
    echo -ne "Usage:\n"
    echo -ne "\t$0 STRING\n"
    echo -ne "\t$0 STRING DIRECTORY\n"
fi
cd ~/
\end{bash}
\end{document}
